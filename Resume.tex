\documentclass[letterpaper,10pt]{article}

\newlength{\outerbordwidth}
\pagestyle{empty}
\raggedbottom
\raggedright
\usepackage[svgnames]{xcolor}
\usepackage{framed}
\usepackage{hyperref}
\usepackage{tocloft}
\usepackage{array}
\definecolor{dgray}{gray}{0.4}
\hypersetup{
  colorlinks=true,
  urlcolor=dgray
}

\usepackage[xetex]{graphicx}
\usepackage{fontspec,xunicode}
\usepackage[
top    = 2.75cm,
bottom = 2.75cm,
left   = 3.00cm,
right  = 2.50cm]{geometry}
\geometry{paperwidth=8.5in, paperheight=13.5in}

\defaultfontfeatures{Mapping=tex-text,Scale=MatchLowercase}


\setlength{\outerbordwidth}{2pt}
\definecolor{shadecolor}{gray}{0.40}
\definecolor{shadecolorB}{gray}{0.90}


\setlength{\evensidemargin}{-0.25in}
\setlength{\headheight}{0in}
\setlength{\headsep}{0in}
\setlength{\oddsidemargin}{-0.25in}
%\setlength{\paperheight}{12in}
%\setlength{\paperwidth}{8.5in}
\setlength{\tabcolsep}{0in}
\setlength{\textheight}{12in}
\setlength{\textwidth}{7in}
%\setlength{\topmargin}{-0.8in}
\setlength{\topskip}{0in}
\setlength{\voffset}{0.1in}


%-----------------------------------------------------------
%Custom commands
\newcommand{\resitem}[1]{\item #1 \vspace{-2pt}}
\newcommand{\resheading}[1]{\vspace{8pt}
  \parbox{\textwidth}{\setlength{\FrameSep}{\outerbordwidth}
    \begin{shaded}
\setlength{\fboxsep}{0pt}\framebox[\textwidth][l]{\setlength{\fboxsep}{4pt}\fcolorbox{shadecolorB}{shadecolorB}{\textbf{\sffamily{\mbox{~}\makebox[6.762in][l]{\large #1} \vphantom{p\^{E}}}}}}
    \end{shaded}
  }\vspace{-5pt}
}
\newcommand{\ressubheading}[4]{
\begin{tabular*}{7in}{l@{\cftdotfill{\cftsecdotsep}\extracolsep{\fill}}r}
\textbf{#1} & #2 \\
\textit{#3} & \textit{#4} \\
\end{tabular*}\vspace{-6pt}}
%-----------------------------------------------------------


\begin{document}

\begin{tabular*}{7in}{l@{\extracolsep{\fill}}r}
\textbf{\Large Agnivo Saha}\\
c/o Mr. Goutam Saha 
&A-515,Lal Bahadur Shah Hall of Residence\\
3rd year Undergraduate Student
& Kharagpur, India, PIN-721302\\
Indian Institute of Technology Kharagpur& +91-9038549169\\
Computer Science and Engineering& \href{mailto:agnivo.saha@gmail.com}{agnivo.saha@gmail.com}
\end{tabular*}
\\[-10pt]


%%%%%%%%%%%%%%%%%%%%%%%%%%%%%%
\resheading{Education}
%%%%%%%%%%%%%%%%%%%%%%%%%%%%%%

\begin{tabular*}{7in}{ c @{\extracolsep{\fill} } c @{\extracolsep{\fill} } c @{\extracolsep{\fill} } r }
\textbf{Degree/Examination}  & \textbf{Institution}  & \textbf{Year}  & \textbf{Grade}   \\
\hline \\ [-1.5ex]
B.Tech(upto $6^{th}$ semester)  & Indian Institute Of Technology,Kharagpur &  2012-2016  & CGPA - 9.67/10  \\
& & & Rank - 4(Total),3(BTech) \\
\hline \\ [-1.5ex]
AISSCE,CBSE Board(Class 12) & Bhavan's Gangabux Kanoria Vidyamandir & 2012 & 95.4\%(Best of 5),97\%(PCM) \\
\hline \\ [-1.5ex]
AISSE,CBSE Board(Class 10) & Bhavan's Gangabux Kanoria Vidyamandir & 2010 & CGPA - 10.0/10.0 \\
\hline \\ [-1.5ex]
\end{tabular*}
\\[-10pt]


%%%%%%%%%%%%%%%%%%%%%%%%%%%%%%
\resheading{Internships}
%%%%%%%%%%%%%%%%%%%%%%%%%%%%%%
\begin{tabular*}{7in}{l@{\extracolsep{\fill}}r}
{\large \textbf{Summer Internship at Microsoft India Development Center}}& May 2015-Jul 2015\\
\end{tabular*}
\begin{itemize}
\setlength{\itemsep}{-4pt}
\item Created a Windows 10 universal Application under the CRM team of Microsoft.
\item Created the design document for the application developed.
\item Used Web Roles (Azure Cloud Service) for the backend service, Azure SQL Database and Azure Blob Storage for the storage of data.
\end{itemize}

{\large \textbf{A Survey on Deep Learning and some of its Applications in Computer Vision}}\\
\begin{tabular*}{7in}{l@{\extracolsep{\fill}}r}
\textbf{Guides: Prof. Sushmita Mitra, Prof. B. Uma Shankar}& June 2014-July 2014\\
\end{tabular*}
\begin{itemize}
\setlength{\itemsep}{-4pt}
\item Implemented Deep Belief Network (using Restricted Boltzmann Machines) for classification of the MNIST dataset.
\item Studied Deep Convolutional Neural Networks and their implementation on the ImageNet Dataset.
\end{itemize}

{\large \textbf{Classification and Clustering of the Iris-dataset}}\\
\begin{tabular*}{7in}{l@{\extracolsep{\fill}}r}
\textbf{Guides: Prof. Sushmita Mitra (I.S.I Kolkata)}& May 2014-June 2014\\
\end{tabular*}
\begin{itemize}
\setlength{\itemsep}{-4pt}
\item Implemented the Multi-layer Perceptron(MLP) for Classifying the Iris data-set.
\item Implemented the Self Organizing Maps/ Kohonen Network for Clustering the Iris data-set into 3 clusters.
\end{itemize}

%%%%%%%%%%%%%%%%%%%%%%%%%%%%%%
\resheading{Academic Projects}
%%%%%%%%%%%%%%%%%%%%%%%%%%%%%%

%{\large \textbf{Summer Internship at Microsoft India Development Center}}\\


{\large \textbf{BTech Project : Deep Learning for NLP}}\\
\begin{tabular*}{7in}{l@{\extracolsep{\fill}}r}
\textbf{Guide: Prof. Sudeshna Sarkar}& Feb 2015-Present\\
\end{tabular*}
\begin{itemize}
\setlength{\itemsep}{-4pt}
\item Calculated the Word Embedding Matrix (word vector representation) using a Neural Network with 1 hidden Layer for a small corpus.
\item Studied Recursive Neural Networks and its application for sentiment analysis and paraphrase detection.
\end{itemize}


{\large \textbf{Information Retrieval Term Project : Implementing a Camera Search Engine}}\\
\begin{tabular*}{7in}{l@{\extracolsep{\fill}}r}
\textbf{Guide: Prof. Sudeshna Sarkar} & Mar 2015-Apr 2015
\end{tabular*}
\begin{itemize}
\setlength{\itemsep}{-4pt}
\item The project was a web based search engine for Nikon Cameras.
\item Crawled Flipkart's website for Nikon Cameras and extracted features such as price, zoom, ratings, reviews.
\item Created an uniform indexing scheme comprising of important feature:value as bigrams and values, features as unigrams and each term was used as index for the cameras.
\item Used nltk library for sentiment analysis on the reviews to predict ratings of Cameras which were missing in Flipkart's website.
\item Calculated weights of the term-camera pair and used cosine-similarity for ranking of results.
\end{itemize}

{\large \textbf{Machine Learning Term Project : Recommendation of Music to Users based on previous ratings}}\\
\begin{tabular*}{7in}{l@{\extracolsep{\fill}}r}
\textbf{Guide: Prof. Sourangshu Bhattacharya} & Oct 2014-Nov 2014
\end{tabular*}
\begin{itemize}
\setlength{\itemsep}{-4pt}
\item Used Yahoo dataset containing user-music ratings.
\item Calculated the top 10 music for an user using Collaborative Filtering (using recsys library) which was used as a baseline for testing.
\item Used PageRank algorithm to predict top 10 music for an user.
\item Defined a Goodness Measure for calculating error between the predicted ranking and the baseline ranking and the sum of the errors for all the users accounted for the accuracy of our algorithm.
\end{itemize}

{\large \textbf{Database Management Term Project : An International Cricket software which supports various types of queries}}\\
\begin{tabular*}{7in}{l@{\extracolsep{\fill}}r}
\textbf{Guide: Prof. Pallab Dasgupta and Prof. Animesh Mukherjee} & Mar 2015-Apr 2015
\end{tabular*}
\begin{itemize}
\setlength{\itemsep}{-4pt}
\item Created the ER diagram for the database design.
\item Enlisted the schemas after Normalization.
\item Crawled espncricinfo website and populated a database with the crawled data and the GUI was made using Java Swing.
\item Supported Range queries, country level statistical queries, ground level statistical queries, tournament level statistical queries, etc.
\end{itemize}

{\large \textbf{Networks Term Project : A peer-to-peer file sharing mechanism using Chord Protocol}}\\
\begin{tabular*}{7in}{l@{\extracolsep{\fill}}r}
\textbf{Guide: Prof. Niloy Ganguly and Prof. Sandip Chakraborty} & Mar 2015-Apr 2015
\end{tabular*}
\begin{itemize}
\setlength{\itemsep}{-4pt}
\item The nodes had the same IP and different Ports and each node had a server running and a client running (who can request for downloading a file).
\item The server maintains a finger table and a list of (file name, IP and port of node in which the file is located) pairs.
\item Implemented search for a file's location (IP and port) using Chord Protocol (using hash of file name and sending through UDP packets).
\item Supported download of file once the connection is established with the node having the file originally.
\end{itemize}

%{\large \textbf{Operating Systems Project : Create a shell}}\\
%\begin{tabular*}{7in}{l@{\extracolsep{\fill}}r}
%\textbf{Guide: Prof. Bivas Mitra} & Feb 2015-Mar 2015
%\end{tabular*}
%\begin{itemize}
%\setlength{\itemsep}{-4pt}
%\item Created the shell which supported basic commands like ls, ls -l, cd, pwd, execution of processes (normal as well as background), cp.
%\item Supported redirection operators (< and >), piping (passing one's output as input to another).
%\end{itemize}

{\large \textbf{Compilers Term Project : Create a Compiler for Tiny C language (a subset of C-language)}}\\
\begin{tabular*}{7in}{l@{\extracolsep{\fill}}r}
\textbf{Guide: Prof. Partha Pratim Das} & August 2014-Nov 2014
\end{tabular*}
\begin{itemize}
\setlength{\itemsep}{-4pt}
\item Used flex for lexical analysis and Used yacc for writing the grammar rules and actions.
\item Supported recursive functions.
\end{itemize}

{\large \textbf{Computer Organization and Architecture Term Project : Create a 32-bit RISC Processor deployed on a SPARTAN 3 FPGA Kit)}}\\
\begin{tabular*}{7in}{l@{\extracolsep{\fill}}r}
\textbf{Guide: Prof. Ajit Pal} & August 2014-Nov 2014
\end{tabular*}
\begin{itemize}
\setlength{\itemsep}{-4pt}
\item Created the design of the RISC processor and the operation codes for the various instructions supported.
\item Used Verilog for implementing the processor.
\item The instructions supported were all Arithmetic operations, branch statements and jump statements.
%\item A program to calculate GCD of two numbers given as input to the processor gave correct output.
\end{itemize}

%{\large \textbf{Software Engineering Project : Create a Medical Laboratory Assistance Software}}\\
%\begin{tabular*}{7in}{l@{\extracolsep{\fill}}r}
%\textbf{Guide: Prof. Partha Pratim Das} & Mar 2014-Apr 2014
%\end{tabular*}
%\begin{itemize}
%\setlength{\itemsep}{-4pt}
%\item Created the UML diagrams(Use Cases, Class, Activity, State, Component, Sequence) for designing the software.
%\item Created the software using Java Swing and performed black box testing.
%\end{itemize}

%{\large \textbf{Bluetooth Based Data Acquisition System}}\\
%\begin{tabular*}{7in}{l@{\extracolsep{\fill}}r}
%\textbf{Guide: Prof. Anjan Rakshit(retd.)} & May 2013-July 2013
%\end{tabular*}
%\begin{itemize}
%\setlength{\itemsep}{-4pt}
%\item Made the circuit containing light and proximity sensors(connected with the SUNROM Bluetooth module) and designed a
% GUI for analyzing the data input.
%\item Was awarded the Second Prize in the JBNSTS Project Presentation.
%\end{itemize}

%%%%%%%%%%%%%%%%%%%%%%%%%%%%%%
\resheading{Academic Achievements and Scholarships}
%%%%%%%%%%%%%%%%%%%%%%%%%%%%%%
\vspace{-8pt}
\begin{itemize}
\setlength{\itemsep}{-2pt}
 \item Received Pre-Placement Offer from Microsoft India Development Center.
 \item Currently placed in the top 1\% of the institute and did a \textbf{Department Change} from Mechanical Engineering with a CGPA of 9.62/10 at the end of the 1st year.
 \item $2^{nd}$ runner up in IBM Hackathon organized in my campus for making a Medical Android application.
 \item Recipient of the \textbf{JBNSTS Scholarship} after qualifying three rounds.
 %\item My team ranked 16th(4th amongst teams of my batchmates) among 50 teams participating in the Catalysts Coding Competition held in my campus in my $2^{nd}$ year.
 \item Awarded eligibility for \textbf{INSPIRE Scholarship} for being in the top 1\% of the CBSE Class 12 Examination.
 \item Secured an All India rank of \textbf{317} in \textbf{Kishore Vaigyanik Protsahan Yojana} Examination in the SX category 
 organized by Department of Science and Technology India.
 \item Secured an \textbf{ALL INDIA RANK OF 1014} in the General Category amongst 4,50,000 candidates in the \textbf{IIT Joint Entrance Examination,2012}
 \item Secured an \textbf{ALL INDIA RANK OF 1404} amongst 11,00,000 candidates in the \textbf{All India Engineering Entrance Exam,2012}.
 \item Secured a rank of \textbf{18} in the West Bengal Joint Entrance Examination,2012.
\end{itemize}

\vspace{-12pt}
%%%%%%%%%%%%%%%%%%%%%%%%%%%%%%
\resheading{Relevant Courses Taken}
%%%%%%%%%%%%%%%%%%%%%%%%%%%%%%
\vspace{-8pt}

\begin{tabbing}
Programming and Data Structures\textsuperscript{*} \hspace{2cm}\= Algorithms-I\textsuperscript{*} \hspace{2cm}\= Discrete Structures \\
Introduction to Electronics\textsuperscript{*} \> Signals and Networks\textsuperscript{*} \> Switching Circuits and Logic Design\textsuperscript{*} \\
Formal Languages and Automata Theory \> Software Engineering\textsuperscript{*} \> Probability and Statistics\\
Computer Organization and Architechture\textsuperscript{*} \> Compilers\textsuperscript{*}\> Algorithms-II\\
Machine Learning \> Matrix Algebra \> Information Retrieval \\
Operating Systems\textsuperscript{*} \> Computer Networks\textsuperscript{*} \> Database Management Systems\textsuperscript{*}\\
Speech and Natural Language Processing\textsuperscript{\#} \> Theory of Computation\textsuperscript{\#}\\
Artificial Intelligence\textsuperscript{\#} \> Advanced Graph Theory\textsuperscript{\#} \> Social Computing\textsuperscript{\#}
\end{tabbing}

\ \\
\vspace{-0.20cm}
\hspace{11.5cm}\footnotesize{
\textsuperscript{*}Both Laboratory and Theory Components}
\\[-5pt]

\ \\
\vspace{-0.20cm}
\hspace{11.5cm}\footnotesize{
\textsuperscript{\#}Currently Ongoing Courses}
\\[-5pt]


\vspace{-10pt}
%%%%%%%%%%%%%%%%%%%%%%%%%%%%%%
\resheading{Skills}
%%%%%%%%%%%%%%%%%%%%%%%%%%%%%%
%\vspace{-2pt}
\large{Programming Languages : C, C++, MySQL(Comfortable), Java, Python, C\#(Novice)

Languages : English, Hindi, Bengali, German(basic).

Platforms : Windows , Linux.}
\vspace{-11pt}

%%%%%%%%%%%%%%%%%%%%%%%%%%%%%%
\resheading{Other Activities}
%%%%%%%%%%%%%%%%%%%%%%%%%%%%%%
\vspace{-4pt}
\begin{itemize}
\setlength{\itemsep}{-2pt}
\item \textbf{National Service Scheme} : Won the Best Volunteer Award for my work in NSS. My NSS unit won the Best Unit Award in NSS Camp.
\item Hobbies include football and Table Tennis.
\end{itemize}

\end{document}
