% (c) 2002 Matthew Boedicker <mboedick@mboedick.org> (original author) http://mboedick.org
% (c) 2003-2007 David J. Grant <davidgrant-at-gmail.com> http://www.davidgrant.ca
% (c) 2008 Nathaniel Johnston <nathaniel@nathanieljohnston.com> http://www.nathanieljohnston.com
%
% (c) 2012 Scott Clark <sc932@cornell.edu> cam.cornell.edu/~sc932
%
%This work is licensed under the Creative Commons Attribution-Noncommercial-Share Alike 2.5 License. To view a copy of this license, visit http://creativecommons.org/licenses/by-nc-sa/2.5/ or send a letter to Creative Commons, 543 Howard Street, 5th Floor, San Francisco, California, 94105, USA.

\documentclass[letterpaper,11pt]{article}
\newlength{\outerbordwidth}
\pagestyle{empty}
\raggedbottom
\raggedright
\usepackage[svgnames]{xcolor}
\usepackage{framed}
\usepackage{hyperref}
\usepackage{tocloft}
\usepackage{array}
\definecolor{dgray}{gray}{0.4}
\hypersetup{
  colorlinks=true,
  urlcolor=dgray
}

\usepackage[xetex]{graphicx}
\usepackage{fontspec,xunicode}
\defaultfontfeatures{Mapping=tex-text,Scale=MatchLowercase}
\setmainfont[Scale=1.0]{Calibri}
%-----------------------------------------------------------
%Edit these values as you see fit

\setlength{\outerbordwidth}{3pt}  % Width of border outside of title bars
\definecolor{shadecolor}{gray}{0.75}  % Outer background color of title bars (0 = black, 1 = white)
\definecolor{shadecolorB}{gray}{0.93}  % Inner background color of title bars


%-----------------------------------------------------------
%Margin setup

\setlength{\evensidemargin}{-0.25in}
\setlength{\headheight}{0in}
\setlength{\headsep}{0in}
\setlength{\oddsidemargin}{-0.25in}
\setlength{\paperheight}{11in}
\setlength{\paperwidth}{8.5in}
\setlength{\tabcolsep}{0in}
\setlength{\textheight}{9.5in}
\setlength{\textwidth}{7in}
\setlength{\topmargin}{-0.3in}
\setlength{\topskip}{0in}
\setlength{\voffset}{0.1in}


%-----------------------------------------------------------
%Custom commands
\newcommand{\resitem}[1]{\item #1 \vspace{-2pt}}
\newcommand{\resheading}[1]{\vspace{8pt}
  \parbox{\textwidth}{\setlength{\FrameSep}{\outerbordwidth}
    \begin{shaded}
\setlength{\fboxsep}{0pt}\framebox[\textwidth][l]{\setlength{\fboxsep}{4pt}\fcolorbox{shadecolorB}{shadecolorB}{\textbf{\sffamily{\mbox{~}\makebox[6.762in][l]{\large #1} \vphantom{p\^{E}}}}}}
    \end{shaded}
  }\vspace{-5pt}
}
\newcommand{\ressubheading}[4]{
\begin{tabular*}{7in}{l@{\cftdotfill{\cftsecdotsep}\extracolsep{\fill}}r}
		\textbf{#1} & #2 \\
		\textit{#3} & \textit{#4} \\
\end{tabular*}\vspace{-6pt}}
%-----------------------------------------------------------


\begin{document}

\begin{tabular*}{7in}{l@{\extracolsep{\fill}}r}
\textbf{\Large Biswajit Paria}  
&Prembazar, P.O. Hijli\\ 
UG (3rd year), Computer Science and Engineering
& Kharagpur, India, PIN-721306\\
Indian Institute of Technology Kharagpur& \href{mailto:biswajitsc@iitkgp.ac.in}{biswajitsc@iitkgp.ac.in}\\
&\href{mailto:biswajitsc@gmail.com}{biswajitsc@gmail.com}
\end{tabular*}
\\[-10pt]


%%%%%%%%%%%%%%%%%%%%%%%%%%%%%%
\resheading{Education}
%%%%%%%%%%%%%%%%%%%%%%%%%%%%%%


\ressubheading{Indian Institute of Technology Kharagpur}{Kharagpur, India}{B.Tech M.Tech Dual Degree (5 year program)}{2012 - 2017 (expected)}

\ \\
Undergraduate 3rd year\\
Current CGPA (Semester 5) : \textbf{9.74/10.00}\\[5pt]
\ressubheading{Kendriya Vidyalaya No. 1, IIT Kharagpur}{Kharagpur, India}{All India Senior School Certificate Examination (CBSE Board)}{2012}

\ \\
Aggregate Score: 92.4\%\\[5pt]
\ressubheading{Kendriya Vidyalaya No. 1, IIT Kharagpur}{Kharagpur, India}{All India Secondary School Examination (CBSE Board)}{2010}

\ \\
CGPA: 9.8/10.0\\[-5pt]


%%%%%%%%%%%%%%%%%%%%%%%%%%%%%%
\resheading{Relevant Courses Taken}
%%%%%%%%%%%%%%%%%%%%%%%%%%%%%%
\begin{tabbing}
Programming and Data Structures \hspace{2cm}\= Algorithms-I \hspace{2cm}\= Discrete Mathematics \\
Introduction to Electronics \> Signals and Systems \> Switching Circuits and Logic Design \\
Formal Languages and Automata Theory \> Software Engineering \> Probability and Statistics\\
Computer Organization and Architechture\textsuperscript{\#}\> Compilers\textsuperscript{\#}\> Algorithms-II\textsuperscript{\#}\\
Computer Organization Laboratory\textsuperscript{\#}\> Compilers Laboratory\textsuperscript{\#}\> Matrix Algebra\textsuperscript{\#}
\end{tabbing}
\vspace{-0.20cm}
\hspace{12.5cm} \footnotesize{
\textsuperscript{\#}To be completed before Dec 2014}\\[-5pt]

%%%%%%%%%%%%%%%%%%%%%%%%%%%%%%
\resheading{Projects and Papers}
%%%%%%%%%%%%%%%%%%%%%%%%%%%%%%

{\large \textbf{A Faceted Recommendation System for Scientific Articles (FeRoSA)}}\\
\begin{tabular*}{7in}{l@{\extracolsep{\fill}}r}
 \textbf{Guides: Prof. Animesh Mukherjee, Prof. Pawan Goyal}&Dec 2013-Mar 2014
\end{tabular*}
\begin{itemize}
\item Developed a Paper Recommendation system based on citation relations and context.
\item Generated citation tags using a Maximum Entropy Classifier on a set of selected features mainly consisting of linguistic features.
\item Extracted topics from the paper texts using an LDA utility.
\item Citation tags were used along with LDA output to generate citation recommendations.
\item Created a GUI for the system.\\[10pt]
\end{itemize}
{\large \textbf{Segmented message broadcast in delay tolerant networks:
An analytical and numerical study}}
\begin{tabular*}{7in}{l@{\extracolsep{\fill}}r}
 \textbf{Guides: Prof. Animesh Mukherjee, Prof. Niloy Ganguly}&May 2013
\end{tabular*}
\begin{itemize}
\item Developed and simulated different Broadcasting strategies on different Network Topologies and measured relevant parameters.
\item Verified asymptotic expressions for expected delays.
\item Computed parameters for optimal Broadcasting.\\[10pt]
\end{itemize}

{\large \textbf{On Farey Table and its Compression for Space Optimization
with Guaranteed Error Bounds}}
\begin{tabular*}{7in}{l@{\extracolsep{\fill}}r}
 \textbf{Guide: Prof. Partha Bhowmick}&May 2012
\end{tabular*}
\begin{itemize}
\item Studied Number-Theoretic properties of Farey Sequences.
\item Developed and implemented compression techniques for the Farey table.
\item Developed mathematical expressions for maximum and average error in compression.
\item Developed mathematical expressions for the number of columns in the compressed table.
\end{itemize}

\pagebreak
%%%%%%%%%%%%%%%%%%%%%%%%%%%%%%
\resheading{Achievements and Scholarships}
%%%%%%%%%%%%%%%%%%%%%%%%%%%%%%
\begin{tabular*}{7in}{l@{\extracolsep{\fill}}r}
\textbf{Overnite-2014 Programming Competition (ACM Certified)}  & \textsc{2014} \\
My team BitBees secured the \textbf{3rd rank} in the onsite finals.\\[5pt]
\textbf{International Online Programming Contest (IOPC), Techkriti- IIT Kanpur}  & \textsc{2014} \\
My team BitBees secured the \textbf{25th rank} among a 1000 international teams.\\[5pt]
\textbf{Sudocode, Kshitij- IIT Kharagpur}  & \textsc{2014} \\
My team BitBees secured the \textbf{1st rank}. Event consisted of three problems based on Algorithms, Image Processing and NLP.\\[5pt]
\textbf{ACM ICPC Regionals Kanpur Site}  & \textsc{2013} \\
My team BitBees secured the \textbf{11th rank} in the online prelims and \textbf{13th rank} in the onsite regionals.\\[5pt]
\textbf{Marauder's Map, Kshitij- IIT Kharagpur}  & \textsc{2013} \\
Got the freshers' prize and \textbf{5th rank} overall in Marauder’s Map, an AI and Image processing event.\\[5pt]
\textbf{Jagadish Bose National Science Talent Search Scholar}  & \textsc{2013} \\
Selected among top \textbf{30} candidates in the state of West Bengal for the scholarship program.\\[5pt]
 \textbf{IIT Joint Entrance Examination (IITJEE)}  & \textsc{2012} \\
All India Rank-\textbf{671} in \textbf{General} category among half a million candidates.\\[5pt]
 \textbf{Indian National Physics Olympiad (INPhO)}  & \textsc{2012} \\
Qualified among \textbf{top 30} candidates to attend the Physics Orientation cum Selection Camp for the International Physics Olympiad.\\[5pt]
 \textbf{Zonal Informatics Olympiad (ZIO)}  & \textsc{2012} \\
Among \textbf{top 253} candidates all over India.\\[5pt]
\textbf{Kishore Vaigyanik Protsahan Yojana (KVPY)}  & \textsc{2011} \\
All India Rank-\textbf{7}. Scholarship Program funded by the Department of Science and Technology, India.\\[5pt]
 \textbf{Indian National Mathematics Olympiad (INMO)} & \textsc{2010} \\
All India Rank-\textbf{6}. Selected to attend the International Mathematics Olympiad Training Camp -2010 and 2011.\\[5pt]
 \textbf{Kendriya Vidyalaya Sangathan (KVS) Junior Mathematics Olympiad} & \textsc{2010} \\
All India Rank-\textbf{2}. Held in all KVS schools.\\[5pt]
\textbf{Australian Mathematics Competition (AMC)- Intermediate Division}  & \textsc{2009} \\
Recieved a Gold Medal along with \textbf{23} other medallists in the world.
\end{tabular*}



%%%%%%%%%%%%%%%%%%%%%%%%%%%%%%
\resheading{Programming Skills}
%%%%%%%%%%%%%%%%%%%%%%%%%%%%%%

\begin{tabular*}{5in}{ll}
{\bf Advanced}& \hspace{1cm}C, C++, \textsc{Java}\\[2pt]
{\bf Basic} &\hspace{1cm}\LaTeX, \textsc{sql, bash, Python, Mathematica}\\[-5pt]
\end{tabular*}

%%%%%%%%%%%%%%%%%%%%%%%%%%%%%%
\resheading{Other Activities}
%%%%%%%%%%%%%%%%%%%%%%%%%%%%%%
Programming Contests, Math Olympiad Teaching and Math Olympiad Interhall competitions, Drawing, Painting
%%%%%%%%%%%%%%%%%%%%%%%%%%%%%%
\resheading{References}
%%%%%%%%%%%%%%%%%%%%%%%%%%%%%%

\begin{tabular*}{7in}{l@{\extracolsep{\fill}}r}
\textbf{Dr. Animesh Mukherjee}  & animeshm@cse.iitkgp.ernet.in\\
Assistant Professor, Dept. of Computer Science and Engineering\\
Indian Institute of Technology Kharagpur\\[4pt]
\textbf{Dr. Niloy Ganguly}  & niloy@cse.iitkgp.ernet.in\\
Associate Professor, Dept. of Computer Science and Engineering\\
Indian Institute of Technology Kharagpur\\[4pt]
\textbf{Dr. Partha Bhowmick}  & pb@cse.iitkgp.ernet.in\\
Associate Professor, Dept. of Computer Science and Engineering\\
Indian Institute of Technology Kharagpur
\end{tabular*}

\end{document}